
\documentclass{stat572Style}

%%\setlength{\oddsidemargin}{0.25in}
%%\setlength{\textwidth}{6in}
%%\setlength{\topmargin}{0.5in}
%%\setlength{\textheight}{9in}

\renewcommand{\baselinestretch}{1.5} 


\bibliographystyle{plainnat}

\begin{document}
%%\maketitle

\begin{center}
  {\LARGE A Review of Lagrangian Time Series Models for Ocean Surface Drifter Trajectories (Sykulski et al. (2016)}\\\ \\
  {Hannah Director \\ 
    Department of Statistics, University of Washington Seattle, WA, 98195, USA
  }
\end{center}



\begin{abstract}
  Put your project summary here.
\end{abstract}

\section{Introduction}
	\subsection{Application}
		\subsection{Data Description}
			Figure 1 included here 
		\subsection{Oceanography background}
			\begin{itemize}
				\item inertial oscillations
				\item turbulent background
			\end{itemize}
			

	\subsection{Spectral analyses of time series}
			\begin{itemize}
				\item  wave equations
				\item Euler's formula
				\item Fourier transformation
				\item periodogram
				\item relationship between autocovariance and power spectral density
				\item complex-valued velocities
				\item include Figure 2
			\end{itemize}

\section{Methods}

	\subsection{Model}
		\subsubsection{Inertial oscillations}
			\begin{itemize}
				\item Ornstein-Uhlenbeck process 
				\item frequency as a free parameter
				\item include figure 3 
			\end{itemize}
		
		\subsubsection{Turbulent background}
				\begin{itemize}
					\item Mat\'{e}rn model 
					\item comparison to other integer order processes (e.g. fractional brownian motion)
				\end{itemize}
		
		\subsubsection{Aggregate model}
			State that you can add two component models together
	
	\subsection{Model fitting}
		\subsubsection{Whittle likelihood}
			\begin{itemize}
				\item explanation of original Whittle likelihood and its problems (aliasing, leakage)
				\item description of tapering `solution' to Whittle and discussion of its imperfections
				\item blurred whittle likelihood 
				\item allows for uncertainty estimates via asymptotics (Fisher information)
			\end{itemize}
	
		\subsubsection{Model misspecification}
			\begin{itemize}
				\item semi-parametric approach in both time and frequency
			\end{itemize}
		\subsubsection{Time-varying parameters for non-stationarity}
		
		\subsubsection{Model selection/likelihood ratio tests}
	



\section{Results}

	\subsection{Simulated results}
		Include Figure 5
		
	\subsection{Real drifter data (with time-varying parameters)}
		Include Figures 6-10

\section{Discussion}
	\begin{itemize}
		\item powerful technique overall
		\item more work needed on selecting windows
	\end{itemize}

\section{Appendix}

\subsection{Errata}
	\begin{itemize}
		\item Typo in equation 13
	\end{itemize}

\subsection{Optimization technique}
	\begin{itemize}
		\item My approach transforming parameters to an unconstrained space gives slightly better (higher maximum likelihood) estimates than their use of Matlab's built-in box constraint approach
			\end{itemize}


\bibliography{stat572}


\end{document}

