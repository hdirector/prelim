
\documentclass{stat572Style}
\usepackage{natbib}
\usepackage{amssymb}
\usepackage{graphicx}
\usepackage{amsmath}
\usepackage{wrapfig}
\usepackage{subfigure}
%%\setlength{\oddsidemargin}{0.25in}
%%\setlength{\textwidth}{6in}
%%\setlength{\topmargin}{0.5in}
%%\setlength{\textheight}{9in}

\renewcommand{\baselinestretch}{1.5} 

\bibliographystyle{plainnat}

\usepackage{color}
\usepackage{ulem}
\newcommand{\vmdel}[1]{\sout{#1}}
\newcommand{\vmadd}[1]{\textbf{\color{red}{#1}}}
\newcommand{\vmcomment}[1]{({\color{blue}{VM's comment:}} \textbf{\color{blue}{#1}})}
\newcommand{\hdcomment}[1]{({\color{red}{HD's comment:}} \textbf{\color{red}{#1}})}

\begin{document}
%%\maketitle

\begin{center}
  {\LARGE A Review of Lagrangian Time Series Models for Ocean Surface Drifter Trajectories (Sykulski et al. (2016))}\\\ \\
  {Hannah Director \\ 
    Department of Statistics, University of Washington Seattle, WA, 98195, USA
  }
\end{center}



\begin{abstract}
  This report reviews the spectral analysis method for modeling ocean surface drifters proposed by \citet{Sykulski2016}. 
  Previous methods to model drifters are discussed along with the authors' model. 
    Where relevant to understanding the model of \citet{Sykulski2016},  spectral analysis, Mat\'{e}rn covariances, the complex-valued Ornstein-Uhlenbeck process, and the Whittle likelihood are reviewed. 
     To conclude the report, we evaluate the strengths and weaknesses of this method. 
  \end{abstract}

\section{Introduction}
\citet{Sykulski2016} presents a multi-component spectral model for analysis of data transmitted by ocean surface drifters. 
Drifters are free-floating  instruments that transmit their location at regular time intervals,  creating a \textbf{\it{Langrangian time series}}, or a sequence of spatial locations over time. 
Figure ~\ref{fig: fig1} shows a sample 250-day drifter trajectory. 
Oceanographers use these time series to increase understanding of  ocean circulation patterns.
 Published in January 2016 in the \textbf{\it{Journal of the Royal Statistical Society Series C}}, this paper develops a model that can isolate a scientific quantity of interest,  and, in so doing, introduces several new statistical techniques.
 \begin{figure}[h!]
	\label{fig: fig1}
  \centering
    \includegraphics[width=.6\textwidth]{ReplicatedFigures/fig1.pdf}
        \caption{A 250-day sample drifter trajectory with two 50-day periods highlighted in green and blue. The time period highlighted in green is not affected by an eddy while the time period highlighted  in blue is.}
\end{figure}
 
\par
The motion of parcels of water moving over changing latitudes is known to have a rotational component. 
Because the Earth has  different diameter at different latitudes, the speed of objects moving with Earth's rotation varies across latitudes.  
This induces a phenomena known as the Coriolis Effect, whereby very fast moving objects or objects being observed over long periods, such as drifters, end up moving at speeds that are different than the Earth beneath them as they change latitude. 
Viewed from a stationary reference frame, this creates circular motions, or inertial oscillations. 
These motions have known frequencies, referred to as inertial frequencies.
 Oceanographers  are interested in detecting deviations from these frequencies, as they are thought to indicate eddies, or persistent circular wave patterns \citep{Kunze1985}.
  However, the ocean has a constant turbulent background flow which makes identifying eddies in real data difficult \citep{ Elipot2010}.
 The majority of the previous work in oceanography has focused on detecting eddies using satellite altimetry rather than drifter data (see \citet{Isern2003} or \citet{Fu2010}, for example), so there is limited work on detecting eddies from drifters.
\citet{Shoosmith2005} and \citet{Lankhorst2006} do identify eddies by finding periods of repeated looping in the drifters' trajectories with a simple autoregressive model.
 Similarly, \citet{Boebel2003} calculate the curvature of the trajectory  and then classify eddies as periods of consistent rotations in one direction. 
 \citet{Sykulski2016} assert that they are the first to identify eddies by explicitly modeling inertial oscillations within a stochastic model.  
 \par
In achieving this result, several statistical advances are made.  
The authors introduce an additive spectral domain model that has components corresponding to both the turbulent background and the inertial oscillations. 
The Mat\'{e}rn covariance commonly used in spatial statistics \citep{Gneiting2012} is applied for the first time to model the turbulent background. 
Because of its flexibility, this covariance structure is shown to have more desirable features than previously-used stochastic models. 
Inertial oscillations are modeled with a complex-valued Ornstein-Uhlenbeck (OU) \citep{Arato1962, Jeffreys1968}, which provides a stochastic equivalent to an accepted set of coupled differential equations describing inertial oscillations. 
Further statistical innovation is employed in fitting the model with a variation on standard Whittle likelihood to reduce bias and in allowing for non-stationarity through time-varying parameter estimates.
\par
The remainder of this report proceeds as follows.
 After a succinct review of spectral analysis, Section 2 discusses the model for the drifters and the methods used to fit it. 
 Section 3 presents the paper's results including analysis of a long time series with time-varying parameters and an application from a simulated model. 
We conclude the paper with a discussion of the value and limitations of the approach taken by \citet{Sykulski2016}. 






\section{Methods}
			

	\subsection{Spectral Analysis}
	\label{sec: specAnalysis}
	\indent We briefly review spectral analysis as all the analysis in \citet{Sykulski2016} is done in the spectral domain.   
	More thorough coverage of this material can be found in  \citet{Percival1993}, for example. 
	From a statistical perspective, a time series, $z(t)$, real or complex-valued,  can be understood as a realization of a stochastic process over time, where  $t = \{1,2,...,n\}$ are the observed time points.  
	Time series are often modeled as the sums of periodic functions, which, using Euler's formula, can be represented compactly as Fourier series. 
	Moreover, data in the time domain can easily be related to data in the frequency domain and vice versa using the Fourier or inverse Fourier transformations 	\begin{align}
z(t) = \int_{-\infty}^{\infty} f_{z}(\omega)e^{i\omega t}d\omega && f_{z}(\omega) = \int_{-\infty}^{\infty} z(t) e^{-i \omega t }dt.
\end{align}
\citep{Percival1993}. 
To understand the distribution of frequencies, the \textbf{\it{power spectral density}},
\begin{equation}
S_{z}(\omega) = \underset{T \rightarrow \infty}{\lim} \mathbb{E} \left(\frac{1}{2T} \left| \int_{-T}^{T} z(t) e^{-i \omega t}dt \right|^{2} \right),
\end{equation}
is often employed. 
This quantity represents the variance (or power) associated with each frequency \citep{Percival1993}. 
Power spectral densities are useful for statistical modeling, since they can be related to the autocovariance sequence of a time series using  Fourier transformation.  
For real $z(t)$, we define the autocovariance as $s_{z}(\tau) = \mathbb{E}[z(t) z(t - \tau)] $ and for complex $z(t)$, we define $s_{z}(\tau) = \mathbb{E}[z(t) z^{*}(t - \tau)] $ where $z^{*}(t)$ represents the complex conjugate of z(t). Letting $\tau$ be the time lag, we have
\begin{align}
\label{eq: fourierPair}
S_{z}(\omega) = \int_{-\infty}^{\infty}s_{z}(\tau) e^{-\omega t}d \tau \Longleftrightarrow s_{z}(\tau) = \frac{1}{2\pi} \int_{-\infty}^{\infty}S_{z}(\omega) e^{i \omega t} d\omega 
\end{align}

\noindent \citep{Sykulski2013}. For an observed time series, the power spectral density is often estimated with the periodogram, $\hat{S}_{z}(\omega) = |J_{z}(\omega)|^{2}$ where 
\begin{equation}
\label{eq: perio}
J_{z}(\omega) = \sqrt{\frac{t}{N}} \sum_{t=1}^{N} z(t) e^{-i \omega t}
\end{equation}
\citep{Sykulski2013}. 
Although intuitively appealing, this estimator has known problems that stem from  sampling a periodic function at discrete intervals. 
In particular,  high frequencies can  capture periodic behavior of other lower frequencies if they happen to align in period. 
Additionally, frequencies above the highest observable frequency cannot be captured, and, are instead, recorded as other frequencies in the spectrum.  
These effects, referred to as $\textbf{\it{aliasing}}$ and $\textbf{\it{leakage}}$ respectively,  are typically reduced by $\textbf{\it{tapering}}$. This method uses  a window of frequencies for estimation rather than a single value. 
However, depending on how tapering is used, it can introduce its own  biases in estimation.  


\subsection{Stochastic Model}
 To model a drifter's movement over time, the eastward and northward components of the drifter's velocity, denoted $u(t)$ and $v(t)$ respectively, is converted to a complex-valued velocity, $z(t) = u(t) + iv(t)$. 
 Separate stochastic models are introduced for the inertial oscillation and turbulent background components of the drifter's behavior. 
 Working in the frequency domain, the two components of the model are added together, creating a six-parameter spectral model. 
 Parameter estimates for this model are obtained via maximizing the model's blurred Whittle likelihood, a relatively new approximation to the true spectral likelihood that balances computational efficiency with minimizing bias \citep{Sykulski2013}. 
 
\subsubsection{Inertial oscillations}
We begin by discussing the model for the inertial oscillations. 
In a deterministic setting, inertial oscillations are often modeled with the following set of coupled differential equations
\begin{align}
\label{eq: deterOsc}
\frac{\partial u }{\partial t}  + f_{0} \nu &= F - cu \\ \nonumber
\frac{\partial v}{\partial t} - f_{0}u &= G - cv
\end{align}
where $u$ and $v$ again represent eastward and northward velocities, $f_{0}$ represents the inertial frequency in radians per unit time, and $F$ and $G$ are forces related to the wind  \citep{Pollard1970}. 
Suppressing the dependence on $t$ and using the complex representation of the velocity,  we show that these relationships can be equivalently expressed as follows
\begin{align}
\label{eq:diffEqDeriv}
\frac{dz}{dt} &= \frac{\partial u}{dt} + i\frac{\partial v(t)}{dt}\\ \nonumber
dz &= (F - c u- f_{0}v)dt + (iG - icv + if_{0}u)dt\\ \nonumber
&= fo(-v + iu)dt - c(u + iv)dt + (F + iG)dt\\ \nonumber
&= if_{0}(u + iv) - c(u + iv) + (F + iG)\\ \nonumber
&= (if_{0} - c)z dt + (F + iG)dt. 
\end{align}
\citet{Sykulski2016} further replace the wind forcing term, $(F + iG)dt$  with complex-valued Brownian increments \citep{Mandelbrot1968} to obtain a stochastic analogue for Equation ~\ref{eq: deterOsc},
 \begin{equation}
\label{eq: ouEq}
dz(t) = (i f_{0} -c) z(t) dt + A d Q(t). 
\end{equation}  
Equation ~\ref{eq: ouEq}  is known as the complex-valued Ornstein-Uhlenbeck (OU) stochastic differential equation. 
In general,  OU processes are considered continuous-time equivalents to the AR(1) process. Specifically,  the univariate OU process is the only mean-zero, stationary, Gaussian, and continuous Markov process, and, similarly the multivariate OU process is the only mean-zero $K$-variate Markov process that is stationary, continuous, Gaussian, and Markov \citep{Schach1971}.
 The complex-valued OU process is the complex representation of a bivariate OU process.  
 In the formulation of Equation ~\ref{eq: ouEq} , $c > 0$ is a dampening parameter that forces the process to always eventually return to its mean position, $z = 0$.  
 The final term of Equation ~\ref{eq: ouEq} is a positive constant, $A$, multiplied by $dQ(t)$.  The term $Q(t), t \geq 0$ is a standard complex Wiener process,  meaning $Q(t) = Q_{1}(t) + i Q_{2}(t)$, where $Q_{1}(t)$ and $Q_{2}(t)$ are standard Wiener processes.
 We remind the reader that Wiener processes are the stochastic models originally used to represent Brownian motion. 
  Such a process $\{Q(t, \omega): t \geq 0\}$ on the measure space $(\omega, Q, P)$ must satisfy
\begin{enumerate}
\item $Q(0, \omega) = 0$ almost everywhere
\item $\{Q(t, \omega): t \geq 0\}$ is distributed normally on $(\Omega,Q, P)$
\item $Q(t + \tau, \omega) - Q(t, \omega)$ has mean 0 and variance $|\tau|$ for all $t, \tau > 0$
\end{enumerate}
\citep{Hida1980}. 

To identify and estimate shifts in the angular frequency, presumably caused by eddies, \citet{Sykulski2016} propose replacing the known inertial requency, $f_{0}$,  in Equation ~\ref{eq: ouEq} with  a free parameter, $\omega_{0}$, to be estimated by the data. 
If the estimated angular frequency differs from the inertial frequency, the angular frequency of the eddy can be calculated as $\omega_{eddy} = f_{0} - \omega_{0}$, since $\omega_{0}$ is a sum of the inertial and eddy frequencies.


For time series modeling, we are interested in the autocovariance of this process at different time lags. To derive this value given as Equation 4 of \citet{Sykulski2016}, we follow \citet{Arato1999} and consider the bivariate rather than complex-valued OU process.
This is possible since the autocovariance of a complex-valued random variable $z(t) = u(t) + i v(t)$ is defined as 
\begin{equation}
\label{eq: expec}
\mathbb{E}\{z(t) z^{*}(t + \tau)\} = \mathbb{E}[u(t)u(t + \tau)] + \mathbb{E}[(v(t) v(t + \tau)] + i\{\mathbb{E}[u(t) v(t + \tau)] + \mathbb{E}[u(t)v( t + \tau)]\}
\end{equation}
\citep{DeIaco2003} where $z^{*}(t + \tau)$ represents the complex conjugate of $z(t + \tau)$. The differential equations for components $u$ and $v$ can be expressed in matrix form as 
\begin{align}
Z(t) = \left( \begin{array}{c} u(t) \\ v(t) \end{array} \right) &= 
\left( \begin{array}{cc} -c & -\omega_{0} \\ \omega_{0} & -c \end{array} \right) \left( \begin{array}{cc} u(t) \\ v(t) \end{array} \right) + \left( \begin{array}{c} A d Q(t) \\ Ad Q(t) \end{array} \right). 
\end{align}
For a multivariate OU process with generator $B$, the covariance matrix for a lag of $\tau$ is $\exp \{B \tau \}C(0) $ where $s(0)$ represents the covariance at zero lag and $B$ is the generator matrix \citep{Schach1971}. 
Using the known variance of a univariate OU process and the independence of the components $u(0)$, we have
\begin{align}
s(0) = \left( \begin{array}{cc} A^{2}/2c & 0 \\ 0 & A^{2}/2c \end{array} \right).
\end{align}
Thus, 
\begin{align}
\label{eq: covDeriv}
\mathbb{E} \left[\left( \begin{array}{c} u(t + \tau) \\ v(t + \tau)  \end{array} \right) \left( \begin{array}{cc} u(t) & v(t) \end{array} \right)  \right] = e^{B \tau} s(0) = 
e^{-c \tau} \left( \begin{array}{cc} \cos \tau \omega_{0} & - \sin \tau \omega \\ \sin \tau \omega_{0} & \cos \tau \omega  \end{array} \right)  \left( \begin{array}{cc} A^{2}/2c & 0 \\ 0 & A^{2}/2c \end{array} \right)
\end{align}
where the second equality can be seen by expanding the infinite series representations for $\exp(-\lambda \tau)$, $\sin(\tau \omega)$, and $\cos( \tau \omega)$. 
Using the relevant components of Equation ~\ref{eq: covDeriv} in Equation ~\ref{eq: expec} and applying Euler's formula, we obtain the complex OU process autocovariance as defined in \citet{Sykulski2016}
\begin{align}
\label{eq: ouAC}
s^{(0)}(\tau) & = \frac{A^{2}}{2c} \exp(i \omega_{0}\tau) \exp(-c|\tau|).
\end{align}
 Using Equation ~\ref{eq: fourierPair}, the corresponding power spectral density can be found to be\begin{align}
\label{eq:ouPSD}
S^{(0)}(\omega) &= \int_{-\infty}^{\infty} s^{(0)}(\tau) \exp (-i \omega \tau) d \tau = \frac{A^{2}}{(\omega - \omega_{0}) + c^{2}}. 
\end{align}
The $(o)$ subscripts denotes that these equations are for the OU component of the full model. 


\subsubsection{Turbulent Background}
The drifters' time series are also marked by the ocean's large-scale turbulence. 
The physics of ocean turbulence is summarized in \citet{Rhines1979}. 
Many models for ocean turbulence have been proposed, including several stochastic models \citep{Lacasce2008}. 
Within the oceanography literature, the stochastic models are designated  first-order \citep{Griffa1995, Falco2000}, second-order \citep{Sawford1991}, and third-order corresponding to  whether the velocity, acceleration, or hyperacceleration is modeled with a Markov process. 
However, calculation of the spectral slopes in observed data indicate that this class of models, referred to as \textbf{\it{integer order}},  does not always adequately reflect the ocean's behavior \citep{Rupolo1996, Sanderson1991}. Rather, \textbf{\it{fractional power models}}, such as fractional Brownian motion (FBM), would better reflect the observed behavior in some cases. 

Towards this end, the authors propose using a Mat\'{e}rn covariance structure \citep{Gneiting2012} for the turbulent background. 
The Mat\'{e}rn model has gained prominence in the spatial statistics literature due to its flexibility.
 In this context, the Mat\'{e}rn model proves useful, since it allows for both integer and fractional powers. 
 The Mat\'{e}rn model is defined either by its autocovariance or power spectral density
\begin{align}
\label{eq:maternAC}
s^{(m)}(\tau) &= \frac{B^{2}}{2^{\alpha - 1/2}\pi^{1/2} \Gamma(\alpha) h^{2 \alpha - 1}}(h|\tau|)^{\alpha - 1}\kappa_{\alpha - 1/2}(h|\tau|)\\
\label{eq:maternPSD}
S^{(m)}(\omega) &= \frac{B^{2}}{(\omega^{2} + h^{2})^{\alpha}}
\end{align}
where $\tau$ again denotes the time lag, $\Gamma(\alpha)$ is the gamma function,  $\kappa_{\eta}$ is a modified Bessel function of the second kind of order $\eta$  and the subscript $(m)$ indicates that this is the Mat\'{e}rn part of the model \citep{Stein2012}. 
The parameters to be estimated are the amplitude, $B$, the dampening, $h > 0$, and a smoothness parameter $\alpha > 1/2$.
 The parameter $\alpha$ is of particular importance, since it controls the smoothness, or degree of differentiability, of the process \citep{Fuentes2010}. 

The authors emphasize that the turbulent background models previously proposed are special cases of the Mat\'{e}rn model. 
This means that these models can still be obtained under the current model if model fitting leads to particular parameter estimate values. 
They also highlight that that Mat\'{e}rn model is more appropriate than the other obvious choice of a fractional power model, FBM, since FBM does not have a bound on the velocities and tends to drift quite far from its mean.
 Figure ~\ref{fig: fmbMat} illustrates these differences by comparing the paths of a simulated particle obtained with a Mat\'{e}rn  covariance versus a FBM covariance. 

\begin{figure}[h!]
	\label{fig: fmbMat}
  \centering
    \includegraphics[width=.7\textwidth]{ReplicatedFigures/fig3.pdf}
        \caption{Simulated velocities obtained using FBM (blue) and Mat\'{e}rn (red)  processes. Top Left: The eastward component the velocities. Top Right: The eastward and northward components of the velocities. Bottom: Periodograms of both processes on the decibel scale. }
\end{figure}

\subsubsection{Overall Model}
Combining the inertial oscillation and turbulent background parts of the model is straightforward.
 To find both the autocovariance and power spectral densities, we can simply add the corresponding components of the inertial oscillation and turbulent background autocovariances. 
 Using Equations ~\ref{eq: ouAC} and ~\ref{eq:maternAC} and Equations ~\ref{eq:ouPSD} and ~\ref{eq:maternPSD}, we obtain
\begin{align}
S(\omega; \boldsymbol{\theta}) &= \frac{A^{2}}{(\omega - \omega_{0}) + c^{2}} + \frac{B^{2}}{(\omega^{2} + h^{2})^{\alpha}}\\
\label{eq: fullSpec}
s_{\tau}(\boldsymbol{\theta}) &= \frac{A^{2}}{2c} \exp(i \omega_{0}\tau) \exp(-c|\tau|) +  \frac{B^{2}}{2^{\alpha - 1/2}\pi^{1/2} \Gamma(\alpha) h^{2 \alpha - 1}}(h|\tau|)^{\alpha - 1}\kappa_{\alpha - 1/2}(h|\tau|).
\end{align}

\begin{figure}[h!]
	\label{fig: fig4}
  \centering
    \includegraphics[width=\textwidth]{ReplicatedFigures/fig4.pdf}
        \caption{Periodograms of the time series sections colored blue and green in Figure ~\ref{fig: fig1}. The black line denotes the inertial frequency.   The dotted lines on the left figure indicates the location of the eddy peak and the expected shift in the inertial frequency. The semi-parametric model fit on the section of the data used in fitting the model is highlighted green on the left figure and blue on the right figure. The red dotted line indicates the model fit extended beyond the frequencies where it was fit. }
\end{figure}


\subsection{Fitting the Model}
\subsubsection{Blurred Whittle likelihood}
\par A recent technique introduced in  \citet{Sykulski2013}  is used to fit the model.
 Referred to as \textbf{\it{blurred Whittle likelihood}}, this new likelihood builds on the well-known Whittle likelihood, a frequency-based approximation  to the time-domain likelihood that reduces computational cost. 
  Whittle likelihood is designed to approximate the likelihood for a univariate Gaussian process, $Z$,  in the time domain that has the likelihood
\begin{align*}
l_{t}(\theta) = - \frac{1}{2} log |C_{Z}(\theta) | - \frac{1}{2} Z^{T} C_{Z}^{-1}(\theta)Z
\end{align*}
where $\theta$ are the model parameters for a particular choice of covariance structure  and $t$ denotes the time domain  \citep{Sykulski2013}. 

 It is known that  the periodogram, given in Equation ~\ref{eq: perio}, satisfies
\begin{equation}
|J_{Z}(\omega)| \overset{d}{\rightarrow} S_{Z}(\omega) \chi^{2}_{2}/2
\end{equation}
under certain regularity conditions \citep{Contreras2006}. 
\citet{Whittle1953} combined this result with Fourier approximations to the true covariance structure and features of Toeplitz matrices to obtain the approximate likelihood
\begin{align}
l_{\omega}(\theta) &= - \sum_{\omega \in \Omega} \left[ J_{Z}^{H}(\omega) \tilde{S}_{Z}^{-1}(\omega; \theta) J_{Z}(\omega) + \log \{S(\omega; \theta) \} \right]\\
\label{eq: bwl}
&= -\sum_{\omega \in \Omega} \left[ \frac{\hat{S}_{Z}(\omega)}{S(\omega;\theta)}  + \log  \{ S(\omega; \theta) \}\right]
\end{align}
 where the subscript $w$ indicates that this is a Whittle likelihood and $H$ denotes a Hermitian matrix. 
  This likelihood can be evaluated in $\mathcal{O}(N log(N))$ where $N$ is the length of the time series which is a notable increase in speed compared to the $\mathcal{O}(N^{3})$  required for time-domain likelihood  \citep{Sykulski2016}. 
  Nevertheless, this likelihood relies on the periodogram and so is affected by aliasing and leakage as discussed in Section ~\ref{sec: specAnalysis}. 
 
To improve on these problems, \citet{Sykulski2013} propose replacing the theoretical spectrum, $S_{Z}(\omega; \theta)$ with the expected value of the spectrum, $\overline{S}_{Z}(\omega, \theta)$\begin{equation}
\overline{S}(\omega; \theta) = \Delta \sum_{\tau = - (N - 1)}^{N-1} \left(1 - \frac{|\tau|}{N} \right)s_{\tau}(\theta) exp ( - i \omega \tau \Delta)
\end{equation}
in Equation ~\ref{eq: bwl}. 
This form of the expected spectrum can be shown to be equivalent to convolving the theoretical spectrum with the Fe\'{j}er kernel 
\begin{equation}
\mathcal{F}(\cdot) = \frac{\Delta}{2\pi N} \frac{sin^{2}(N \omega \Delta/2)}{sin^{2}(\omega \Delta /2)} 
\end{equation} \citep{Sykulski2013}. 
This improves the estimators obtained through maximizing the likelihood, because the expected value of the periodogram $\mathbb{E}[|J_{Z}(\omega)^{2}]$ is blurred in the same way as the observed periodogram. Thereby, effects due to sampling at discrete intervals are avoided. 
This leads to the final likelihood
\begin{align}
l_{b}(\theta) = - \sum_{\omega \in \Omega} \left[\frac{\hat{S}_{Z}(\omega)}{\overline{S}(\omega; \theta)} + \log \{ \overline{S} (\omega; \theta) \}\right]
\end{align}
which is maximized by standard optimization techniques. 

Using asymptotic properties of the Whittle likelihood, we also have 
\begin{equation}
\label{eq: fisher}
\sqrt{N}(\hat{\theta} - \theta) \overset{d}{\rightarrow} \text{ Normal($0, \mathcal{F}^{-1})$ }.
\end{equation}
where $\mathcal{F}$ is the Fisher information. 
Thus, uncertainty estimates of the parameters can easily be obtained using numerical calculation of the Hessian. 

\subsubsection{Semi-parametric fitting}
\label{sec: semi}
The authors note that the fitting of Equation ~\ref{eq: fullSpec} can be adapted to reflect background knowledge about the drifters. 
They refer to making these adjustments as applying a semi-parametric approach. 
In their analysis, they highlight using knowledge of the physical system and adjusting for artifacts from interpolation. 
Specifically,  variability can be different for the eastward versus northward velocities  which creates spurious correlations between positive and negative correlations. 
To avoid this, the authors only fit the model using data from either the positive or negative half of the spectrum depending on the sign of the inertial oscillation, which is known based on the latitude of the drifter.  
Similarly, sampling effects make the accuracy of high frequencies suspect, so the authors only use frequencies up to a specified threshold in fitting the model. Figure ~\ref{fig: fig4} illustrates a semi-parametric model fit. 

\subsubsection{Non-stationarity}
Further, to allow for non-stationarity in long time series, \citep{Sykulski2016} propose fitting the model independently at each observed time point using a rolling window around the time point of interest. 
This gives the OU differential equation
\begin{equation}
dz(t) = \{i \omega_{0}(t) - c(t) \} z(t) dt + A(t) dQ(t)
\end{equation}
and spectral power density
\begin{equation}
S^{(0)}(\omega, t) = \frac{A(t)^{2}}{\{\omega - \omega_{0}\}^{2} + c(t)^{2}}. 
\end{equation}
with parameters now a function of time. 
Similarly, the time-varying version of Mat\'{e}rn power spectral density is 
\begin{equation}
S^{(m)}(\omega, t) = \frac{B^{2}(t)}{\{\omega^{2} + h^{2}(t)\}^{\alpha(t)}}.
\end{equation}
The length of the window to fit these time-varying parameters needs to be large enough to allow for a full sampling of the spectrum, but short enough to be treated as locally stationary. 
 In this analysis, the authors select a reasonable window length empirically, but acknowledge that further work on this subject would be of value.  

\subsubsection{Significance tests for simpler models}
\label{sec: sig}
Finally, \citet{Sykulski2016} propose that a simple likelihood ratio test can be used at each time period to test whether or not a shift in the inertial oscillation has occurred. 
To do so, the authors suggest fitting a simpler model where the the parameter $\omega_{0}$ is fixed to be the inertial frequency $f$. 
Then this simpler model can be compared to the full six parameter model via a likelihood ratio test with test statistic
\begin{equation}
\label{eq: LRT}
R(t) = 2[l_{b} \{\hat{\theta}_{1}(t) \} - l_{b}\{\hat{\theta}_{0}(t) \} ]
\end{equation}
where $\hat{\theta}_{0}(t)$ is the fitted parameter vector for the simpler, or null model, and $\hat{\theta}_{1}(t)$ is the fitted parameter vector for the full model.
 The authors claim $R(t)$ then follow a Chi-squared distribution with 1 degree of freedom. 
 However, we note that this may be overly simplistic, because of potential identifiability issues with the parameters and lack of independence between time points stemming from the use of a rolling window. 
 

\section{Results}
We show the potential utility of this method by analyzing two non-trivial data sets.  
First, we consider the modeled trajectories of 200 near-surface particles obtained from a computer simulation of ocean circulation. 
The simulation is set to be `very similar' to the SPEM 5.2 set-up of \citet{Danioux2008}). 
In Figure ~\ref{fig: numSim}, we display the fit of the aggregate spectral model to a single trajectory from the simulation and to an ensemble of trajectories. 
The resulting spectra has peaks at zero and at the inertial frequency, which indicates that the model captures the turbulent background and the inertial oscillations, which are set not to have shifts. 
We also display the individual periodograms and model fits. 
\citet{Sykulski2016}  note that this highlights the method's robustness, since the individual periodograms show considerable variability, but the model fits are all quite consistent.

\begin{figure}[h!]
	\label{fig: numSim}
  \centering
    \includegraphics[width=\textwidth]{ReplicatedFigures/fig5.png}
        \caption{Left: The spectral density of one particle (blue) with the model fit overlaid for the portion of the spectra used to fit it (green) and extended to all frequencies (red). Right: The ensemble mean periodogram (blue) with the ensemble fit for the frequencies used to fit it (green) and extended to the entire spectrum (red) and individual trajectories' periodograms (light grey) and model fits (dark fit) }
\end{figure}

We also consider the trajectory of a drifter in the South Pacific Ocean observed 12 times a day over a 1642 day period. 
Using a rolling window of 1000 observations and the semi-parameteric set-up described in Section ~\ref{sec: semi} , we obtain parameter estimates for the spectral model at each time point.
 In Figure ~\ref{fig:timeVarying}, we compare the observed spectral density over time and the fitted spectral density.
  The fitted density appears to be a smoothed version of the observed density, indicating a reasonable model fit.
   We also display the fitted parameter values over time and their 95$\%$ confidence interval bands in Figure ~\ref{fig: fig8} report the correlation among variables in Figure 10 (and fig and renumber). Figure ~\ref{fig: timeVarying} displays the observed spectra versus the modeled spectra. 

We also fit a 5-parameter where the frequency is set to the inertial frequency rather than being a free-parameter. 
We then compare the fit to the full model using a likelihood ratio test as described in Section ~\ref{sec: sig}.
 In Figure 9, we display the test statistic over time where a red line is used to indicate the level of statistical significance. 
 We conclude for time periods where the test statistic is greater than this cut-off that there is a shift in the inertial frequency. 

\begin{figure}[hb]
	\label{fig:timeVarying}
  \centering
    \includegraphics[width=\textwidth]{ReplicatedFigures/fig7.png}
        \caption{Top: Observed spectra of the drifter over time; Bottom: Modeled spectra over time. On both figures, the white line indicates the average inertial frequency within the rolling window at the current time period.  Only frequencies used in the estimation are included.  }
\end{figure}

 \begin{figure}[h!]
	\label{fig: fig9}
  \centering
    \includegraphics[width=\textwidth]{ReplicatedFigures/fig9.pdf}
        \caption{ADD CAPTION}
\end{figure}

\begin{figure}[h!]
	\label{fig: fig8}
  \centering
    \includegraphics[width=\textwidth]{ReplicatedFigures/fig8.pdf}
        \caption{ADD CAPTION}
\end{figure}



\section{Discussion}
identifiability 



\clearpage

\bibliography{prelim}


\section{Appendix}

\subsection{Errata}

\subsection{Additional Figures}

\begin{figure}[h!]
	\label{fig: fig2}
  \centering
    \includegraphics[width=\textwidth]{ReplicatedFigures/fig2.pdf}
        \caption{ADD CAPTION}
\end{figure}




\subsection{Computation}



\end{document}

