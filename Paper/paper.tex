
\documentclass{stat572Style}
\usepackage{natbib}
\usepackage{amssymb}
%%\setlength{\oddsidemargin}{0.25in}
%%\setlength{\textwidth}{6in}
%%\setlength{\topmargin}{0.5in}
%%\setlength{\textheight}{9in}

\renewcommand{\baselinestretch}{1.5} 

\bibliographystyle{plainnat}

\begin{document}
%%\maketitle

\begin{center}
  {\LARGE A Review of Lagrangian Time Series Models for Ocean Surface Drifter Trajectories (Sykulski et al. (2016)}\\\ \\
  {Hannah Director \\ 
    Department of Statistics, University of Washington Seattle, WA, 98195, USA
  }
\end{center}



\begin{abstract}
  Put your project summary here.
\end{abstract}

\section{Introduction}
	\subsection{Application}
		\subsection{Data Description}
			Figure 1 included here 
		\subsection{Oceanography background}
			\begin{itemize}
				\item inertial oscillations
				\item turbulent background
			\end{itemize}
			

	\subsection{Spectral analyses of time series}
	TO DO: vary who I cite here, unify notation across sources\\
	\indent \citet{Sykulski2016} exclusively employs spectral analysis to model the  drifters.  To provide context for the  methods they propose, this section summarize the main concepts of this time series analysis method. More thorough coverage of this material can be found in  \citep{Percival1993}, for example. From a statistical perspective, a time series, $X_{t} = X(t\Delta)$, where $\Delta > 0 $ is the sampling interval and $t = \{1,2,...,n\}$,  can be understood as a realization of a stochastic process over time. Spectral analysis models time series, not in their original time domain, but instead in a frequency domain. This is possible because time series can be represented as infinite sums of exponential functions, known as a spectral representation. Following the presentation of \citet{Percival1993}, the idea of spectral representation can be understood by initially considering the harmonic process,
    	\begin{equation}
		\label{eq: harmonic}
		X_{t} = \sum_{l=1}^{L}D_{l} cos(2\pi f_{l}t + \phi_{l}),  t = 0, \pm 1, \pm 2...,
	\end{equation}
where $L$ and  $D_{l}$, and $f_{l}$ are real-valued constants, $\phi_{l}$ are independent Uniform($-\pi, \pi$) random variables, and the frequencies, $0 <f_{l} < 1/2$, are taken to be increasing order. As the number of  sinusoidal functions increases,  more complex behavior can be represented by the sum in Equation ~\ref{eq: harmonic}.  Applying Euler's formula,  Equation \ref{eq: harmonic}, can be written as 
	\begin{equation}
		\label{eq: harmonicSpec}
		X_{t} = \sum_{l = -L}^{L} C_{l}e^{i2\pi f_{l} t}.
	\end{equation}
	
\noindent which as $L \rightarrow \infty$  gives
	\begin{equation}
		\label{eq: specRep}
		X_{t} = \int_{-1/2}^{1/2} e^{i2\pi ft}dZ(f),
	\end{equation}

\noindent Equation \ref{eq: specRep} can be used to define any real or complex-valued discrete parameter stationary process, $X_{t}$, with mean zero. Specifically, there exists a process $\{Z(f)\}$ on $[-1/2, 1/2]$,  where the intervals $dZ(f)$ and $dZ(f')$  are uncorrelated for all $f,f'$, that satisfies Equation \ref{eq: specRep} for all $t$\citep{Percival1993}. This result is known as the spectral representation theorem for discrete parameter stationary processes. An analogous result holds for continuous parameter stationary processes \citep{Percival1993}. 

The relationship between spectral densities and autocovariances also contributes to the utility of analysis in this domain. In particular, the spectral densities of a time series and the autocovariance sequence are the Fourier transform of one another.  In this paper, the focus is on second order stationary processes,  or cases where the autocovariance, $s_{xx}$, is only a function of the distance between the increments,  meaning $c_{xx}(t_{1}, t_{2}) = \mathbb{E}(X_{t1}X_{t2}) = s_{xx}(|t_{1} - t_{2}|)$.  (add derivation here?) In such cases, this relationship is written as 
	\begin{equation}
	S_{xx}(\omega) = \Delta \sum_{\tau = -\infty}^{\infty} s_{xx}(\tau)e^{i\omega \tau \Delta} 
	\end{equation}
and 
	\begin{equation}
		s_{xx}(\tau) = \frac{1}{2\pi} \int_{-\pi / \Delta}^{\pi / \Delta}S_{xx}(\omega)e^{i \omega \Delta}d \omega.
	\end{equation}
\noindent where $\omega \in [-\pi/\Delta, \pi/ \Delta]$ is the angular frequency measured in radians \citep{Sykulski2013}
.  In the complex-valued case, $s_{xx}$ and $S_{xx}$ are replaced with $s_{xy}$ and $S_{xy}$ to represent the cross-covariance terms \citep{Sykulski2013}. 

Given an observed time series $X_{t}$, $t = 1,2,..,N$, the common estimator for the spectra is the periodogram, $\hat{S}_{xx}(\omega) = |J_{x}(\omega)|^{2}$ where 
\begin{equation}
J_{x}(\omega) = \sqrt{\frac{\Delta}{N}} \sum_{t=1}^{N} X_{t}e^{-i \omega t \Delta}. 
\end{equation}
Although simple, this estimator has two known issues: aliasing and leakage.  Aliasing is ... Leakage is... Tapering has been proposed to fix, but has problems.. 


\section{Methods}

	\subsection{Model}
		\subsubsection{Inertial oscillations}
			\begin{itemize}
				\item Ornstein-Uhlenbeck process 
				\item frequency as a free parameter
				\item include figure 3 
			\end{itemize}
		
		\subsubsection{Turbulent background}
				\begin{itemize}
					\item Mat\'{e}rn model 
					\item comparison to other integer order processes (e.g. fractional brownian motion)
				\end{itemize}
		
		\subsubsection{Aggregate model}
			State that you can add two component models together
	
	\subsection{Model fitting}
		\subsubsection{Whittle likelihood}
				
			\begin{itemize}
				\item explanation of original Whittle likelihood and its problems (aliasing, leakage)
				\item description of tapering `solution' to Whittle and discussion of its imperfections
				\item blurred whittle likelihood 
				\item allows for uncertainty estimates via asymptotics (Fisher information)
			\end{itemize}
	
		\subsubsection{Model misspecification}
			\begin{itemize}
				\item semi-parametric approach in both time and frequency
			\end{itemize}
		\subsubsection{Time-varying parameters for non-stationarity}
		
		\subsubsection{Model selection/likelihood ratio tests}
	



\section{Results}

	\subsection{Simulated results}
		Include Figure 5
		
	\subsection{Real drifter data (with time-varying parameters)}
		Include Figures 6-10

\section{Discussion}
	\begin{itemize}
		\item powerful technique overall
		\item more work needed on selecting windows
	\end{itemize}

\section{Appendix}

\subsection{Errata}
	\begin{itemize}
		\item Typo in equation 13
	\end{itemize}

\subsection{Optimization technique}
	\begin{itemize}
		\item My approach transforming parameters to an unconstrained space gives slightly better (higher maximum likelihood) estimates than their use of Matlab's built-in box constraint approach
			\end{itemize}


\bibliography{prelim}


\end{document}

